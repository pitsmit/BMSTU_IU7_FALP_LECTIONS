\chapter{Анализ предметной области}

В данном разделе вводятся основные определения, связанные с временными рядами, определяются задачи
их хранения и прогнозирования.

\section{Понятие временного ряда}

Временной ряд — это последовательность упорядоченных во времени числовых показателей, характеризующих 
уровень состояния и изменения изучаемого явления [1].

Временной ряд обязательно включает два элемента: время и конкретное значение показателя — уровень
ряда. Показатель времени может быть представлен как конкретные моменты времени, либо 
как отдельные временные периоды. 

Длина ряда — количество входящих в него уровней $n$.
Ряд принято обозначать как $Y_{t}$, где $t = 1, \ldots, n$.

На рисунке \ref{fig:row_example} приведён пример временного ряда, отражающего цену акции на бирже.

\begin{figure}[H]
	\centering
	\includegraphics[width=0.8\textwidth]{img/row_example.png}
	\caption{Пример временного ряда — цена акции на бирже.}
	\label{fig:row_example}
\end{figure}

\section{Способы классификации временных рядов}

\subsection{По времени}

При таком способе классификации различают моментные и интервальные ряды. 

Интервальный ряд — последовательность, в которой уровень явления относят к результату, накопленному или вновь 
произведенному за определенный интервал времени.

Моментный ряд — последовательность, в которой уровень характеризует изучаемое явление в 
конкретный момент времени.

\subsection{По форме представления}

Данная классификация подразделяет временные ряды на ряды абсолютных, относительных и средних
величин. Она помогает выбирать исследователям подходящие методы анализа в зависимости от специфики 
данных и целей исследования.

\subsection{По расстоянию между показателями времени}

Выделяют полные и неполные временные ряды. Полные ряды применяются, когда даты регистрации или 
окончания периодов следуют друг за другом с равными интервалами. Неполные, в свою очередь, наоборот,
когда принцип равных интервалов не соблюдается. 

\subsection{По содержанию показателей}

По содержанию показателей ряды делятся на ряды частных и агрегированных показателей.
Частные показатели характеризуют один конкретный частный признак изучаемого явления. Агре­гированные 
показатели основаны на частных и характеризуют изу­чаемое явление комплексно.

\section{Задача хранения временных рядов}

Часто ценность данных временного ряда возрастает с увеличением срока их давности — данные, 
получаемые в режиме реального времени, оказываются менее полезными, чем собранные много лет назад [2]. 
Для возможности обработки временных рядов в будущем, необходимо решить задачу об их хранении.

Правильное хранилище данных должно обладать следующими характеристиками:

\begin{itemize}[label=—]
	\item высокой надёжностью;
	\item простотой доступа к данным;
	\item нетребовательностью к вычислительному оборудованию.
\end{itemize}

Разработка универсального решения, отвечающего за хранение временных рядов, является сложной задачей, 
поскольку они бывают разных видов, каждый из которых имеет свой формат хранения, чтения, записи 
и обработки данных.

\section{Задача прогнозирования временных рядов}

Формально задача прогнозирования сводится к получению оценок значений ряда для некоторого периода 
будущего, т.е. к получению значений $Y_{t}$, где $t = n + 1, n + 2, \ldots$.

Наиболее тривиально задача решается при допущении, что в предыдущие периоды времени не происходило 
существенных изменений исследуемой характеристики или если изменения носили разнонаправленный характер 
и взаимно погашались. В этом случае прогнозное значение принимается равным предыдущему фактическому 
значению, такой прогноз называют прогнозом без изменений:
\begin{equation}
	y_{t} = y_{t-1}
\end{equation}
где $y_{t-1}$ — фактическое значение в момент времени $t - 1$.