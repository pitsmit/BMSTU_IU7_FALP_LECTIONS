\chapter{Лекция 5 — 10 марта 2025}

\section{Функционалы}

\begin{listbox}{\noindent \begin{listboxtitle}{}2\end{listboxtitle} 
	\raisebox{6pt}{Функционалы делятся на два типа:}}
\begin{enumerate}
	\item отображающие -- применяют функцию многократно к 
    элементам списка, имя часто начинается на map;
	\item применяющие -- apply, funcall.
\end{enumerate}
\end{listbox}

\section{apply}

APPLY принимает 2 аргумента, из которых первый аргумент представляет 
собой функцию, которая применяется к элементам списка, составляющим 
второй аргумент функции. Должно быть выдержано соответствие между
структурой элементов списка и функцией.  Функция может быть
определена через lambda. Символ \# называется 
функциональной блокировкой. Необходим для явного указанию 
интерпретатору, что первый аргумент APPLY -- функция, а не переменная.
В противном случае возникнет ошибка несвязанной переменной.

\begin{figure}[H]
    \begin{listingbox}{}
        \lstinputlisting[language=Lisp]{lists/apply.lisp}
    \end{listingbox}
    \caption{Пример работы функции apply}
    \label{lst:apply-example}
\end{figure}

\section{funcall}

FUNCALL по своему действию аналогичен APPLY, но аргументы для вызываемой 
функции он принимает не списком, а по отдельности.

\begin{figure}[H]
    \begin{listingbox}{}
        \lstinputlisting[language=Lisp]{lists/funcall.lisp}
    \end{listingbox}
    \caption{Пример работы функции funcall}
    \label{lst:funcall-example}
\end{figure}

\section{mapcar}

MAPCAR возвращает список, полученный из результатов применения
функции к элементам списка. Список формируется с помощью LIST.
Количество аргументов у функции должно соответствовать количеству 
передаваемых в MAPCAR списков. На каждом шаге работы MAPCAR выделяет
i-е элементы списков и передаёт в функцию. Если передаваемые 
списки разной длины, на выходе получится список длины минимального 
списка. 

\begin{figure}[H]
    \begin{listingbox}{}
        \lstinputlisting[language=Lisp]{lists/mapcar.lisp}
    \end{listingbox}
    \caption{Пример работы функции mapcar}
    \label{lst:mapcar-example}
\end{figure}

\section{maplist}

MAPLIST применяет переданную функцию сначала ко всему списку,
потом к нему без первого элемента и т.д. Список формируется с помощью 
LIST. Работа MAPLIST с несколькими списками аналогична MAPCAR, только
возвращается список списков.

\begin{figure}[H]
    \begin{listingbox}{}
        \lstinputlisting[language=Lisp]{lists/maplist.lisp}
    \end{listingbox}
    \caption{Пример работы функции maplist}
    \label{lst:maplist-example}
\end{figure}

\section{mapcan, mapcon}

MAPCAN и MAPCON аналогичны MAPCAR и MAPLIST соответственно, за 
исключением того, что результат формируется с помощью функции 
NCONC, что разрушает искомый список.

\section{find-if}

FIND-IF применяет функцию к car элементам списка, возвращает
первый car элемент, удовлетворивший условию. В противном случае
возврат NIL.

\begin{figure}[H]
    \begin{listingbox}{}
        \lstinputlisting[language=Lisp]{lists/find-if.lisp}
    \end{listingbox}
    \caption{Пример работы функции find-if}
    \label{lst:find-if-example}
\end{figure}

\section{remove-if, remove-if-not}

REMOVE-IF возвращает новый список с элементами,
для которых предикат вернул NIL.

REMOVE-IF-NOT возвращает новый список с элементами,
для которых предикат вернул T.

\begin{figure}[H]
    \begin{listingbox}{}
        \lstinputlisting[language=Lisp]{lists/remove-if-remove-if-not.lisp}
    \end{listingbox}
    \caption{Пример работы функций remove-if, remove-if-not}
    \label{lst:remove-if-remove-if-not-example}
\end{figure}

\section{reduce}

REDUCE каскадно преобразует элементы списка, используя
двухаргументную функцию. Сначала к первому и второму, потом к 
образовавшемуся элементу и 3 элементу списка и т.д.

\begin{figure}[H]
    \begin{listingbox}{}
        \lstinputlisting[language=Lisp]{lists/reduce.lisp}
    \end{listingbox}
    \caption{Пример работы функции reduce}
    \label{lst:reduce-example}
\end{figure}

\section{every, some}

EVERY применяет предикат к каждому элементу списка. Если предикат
возвращает NIL, EVERY возвращает NIL. Если дошла до конца
списка, то возвращается T.

SOME применяет предикат к элементам до тех пор, пока он не вернёт
T. Если дошла до конца списка, то возвращает NIL.

По аналогии с MAPCAR можно передавать сразу несколько списков
в EVERY и SOME.

\begin{figure}[H]
    \begin{listingbox}{}
        \lstinputlisting[language=Lisp]{lists/every-some.lisp}
    \end{listingbox}
    \caption{Пример работы функций every, some}
    \label{lst:every-some-example}
\end{figure}