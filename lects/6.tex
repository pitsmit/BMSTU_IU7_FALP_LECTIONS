\chapter{Лекция 6 — 17 марта 2025}

\section{Основные понятия о рекурсии}

\begin{listbox}{\noindent \begin{listboxtitle}{}2\end{listboxtitle} 
    \raisebox{6pt}{Методы организации повторных вычислений:}}
\begin{enumerate}
	\item функционалы;
	\item рекурсия.
\end{enumerate}
\end{listbox}

\begin{listbox}{\noindent \begin{listboxtitle}{}3\end{listboxtitle} 
    \raisebox{6pt}{Классификация рекурсии:}}
\begin{enumerate}
	\item простая -- один рекурсивный вызов;
	\item первого порядка -- несколько рекурсивных вызовов;
	\item взаимная -- функции рекурсивно вызывают друг друга.
\end{enumerate}
\end{listbox}

\begin{listbox}{\noindent \begin{listboxtitle}{}2\end{listboxtitle} 
    \raisebox{6pt}{Классификация рекурсии в рамках классификации:}}
\begin{enumerate}
	\item хвостовая -- рекурсия, при завершении которой результат УЖЕ 
    сформирован, не отстается отложенных вычислений;
	\item дополняемая -- при обращении к рекурсии используется дополнительная
    функция ВНЕ рекурсивного вызова;
    \item множественная -- на одной ветке несколько рекурсивных вызовов.
\end{enumerate}
\end{listbox}

\begin{listbox}{\noindent \begin{listboxtitle}{}3\end{listboxtitle} 
    \raisebox{6pt}{Проблемы рекурсии:}}
\begin{enumerate}
	\item как совершить первый вызов;
	\item как остановить выполнение;
	\item как организовать новый шаг.
\end{enumerate}
\end{listbox}

\section{Примеры построения рекурсий разных видов}

\begin{figure}[H]
    \begin{listingbox}{}
        \lstinputlisting[language=Lisp]{lists/recursion-add.lisp}
    \end{listingbox}
    \caption{Пример дополняемой рекурсии}
    \label{lst:recursion-add-example}
\end{figure}

\begin{figure}[H]
    \begin{listingbox}{}
        \lstinputlisting[language=Lisp]{lists/recursion-tail.lisp}
    \end{listingbox}
    \caption{Пример хвостовой рекурсии}
    \label{lst:recursion-tail-example}
\end{figure}

\begin{figure}[H]
    \begin{listingbox}{}
        \lstinputlisting[language=Lisp]{lists/recursion-many.lisp}
    \end{listingbox}
    \caption{Пример множественной рекурсии}
    \label{lst:recursion-many-example}
\end{figure}

\begin{figure}[H]
    \begin{listingbox}{}
        \lstinputlisting[language=Lisp]{lists/recursion-sort.lisp}
    \end{listingbox}
    \caption{Пример сортировки вставками с помощью рекурсии}
    \label{lst:recursion-sort-example}
\end{figure}