\chapter{Существующие методы решения}

В данном разделе описываются основные методы хранения и прогнозирования временных рядов.

\section{Методы хранения временных рядов}

Существует несколько способов организации хранилищ данных временных рядов, каждый из которых
основывается на определённом техническом средстве, среди которых наиболее распространёнными являются:

\begin{itemize}[label=—]
	\item реляционная база данных;
	\item нереляционная база данных;
	\item специализированная база данных временных рядов;
	\item плоские файлы.
\end{itemize}

При выборе способа хранения данных необходимо учитывать ключевые требования к средствам хранения 
временных рядов:

\begin{itemize}[label=—]
	\item операции записи преобладают над операциями чтения;
	\item данные записываются, считываются и обновляются во временном порядке;
	\item одновременное считывание данных выполняется намного чаще, чем при работе с транзакциями;
	\item массовое удаление данных выполняется чаще, чем удаление отдельных точек данных.
\end{itemize}

\subsection{Преимущества хранения временных рядов в реляционных базах данных}

Хранение временных рядов в реляционных базах имеет множество преимуществ, среди которых:

\begin{itemize}[label=—]
	\item временной ряд можно легко связать с соответствующими перекрестными данными этой же базы данных;
	\item иерархические данные временных рядов естественным образом встраиваются в реляционные таблицы;
	\item при создании временных рядов на основе транзакционных данных упрощается их дальнейшая проверка
	и наполнение перекрёстными связями.
\end{itemize}

\subsection{Преимущества хранения временных рядов в нереляционных базах данных}

Способ хранения временных рядов в нереляционных базах обладает рядом преимуществ, среди которых:

\begin{itemize}[label=—]
	\item высокая скорость записи;
	\item удобство разработки функциональных и высокопроизводительных решений в системах, 
	в которых мало что известно о будущих данных;
	\item низкая вероятность образования малопригодной производственной схемы на основе
	нереляционной базы данных.
\end{itemize}

\subsection{Сравнение скорости работы со временными рядами при использовании различных баз данных}

Стоит упомянуть, что все специализированные базы данных временных рядов основываются на модели 
хранения в нереляционных базах данных. Поэтому скорость работы операций в этих базах примерно равна и 
ведёт себя одинаково при разрастании объёма хранимых данных.

На графике \ref{fig:speed_graph} приведено сравнение производительностей операции записи новых точек 
временных рядов в стандартную реляционную базу данных и базу данных временных рядов.

\begin{figure}[H]
	\centering
	\includegraphics[width=0.8\textwidth]{img/speed_graph.png}
	\caption{Сравнение скорости вставки данных в реляционную базу данных и базу данных временных рядов. }
	\label{fig:speed_graph}
\end{figure}

\subsection{Файловые решения}

Хранение временных рядов на основе плоских файлов широко распространено в научных программах и 
различных типах промышленного программного обеспечения, в которых производительность имеет 
первостепенное значение. В таких ситуациях разработка конвейера данных, обеспечивающего распределение 
памяти для хранения данных, открытие, чтение, закрытие файлов и их защита выполняются вручную. 

Реализация хранения временных рядов может оказаться самым эффективным решением, поскольку обладает 
несомненными преимуществами по сравнению с решениями, рассмотренными выше:

\begin{itemize}[label=—]
	\item формат плоского файла не зависит от системы;
	\item временные затраты на операции ввода и вывода данных в плоском файле гораздо меньше по 
сравнению со всеми видами баз данных;
	\item порядок, в котором должны считываться данные, указывается в самом формате файла;
	\item благодаря точечной настройке степени сжатия данных, конечные данные могут занимать в памяти
гораздо меньший объём, чем при использовании баз данных.
\end{itemize}

\section{Методы прогнозирования временных рядов}

Большинство методов прогнозирования основаны на экстраполяции [3]. Экстраполяция — это метод научного 
исследования, предполагающий распространение прошлых и настоящих тенденций, закономерностей, 
связей на будущее развитие объекта прогнозирования. Ниже рассмотрены некоторые из методов прогнозирования.

\subsection{Прогнозирование по среднему уровню ряда}

Данный метод подходит для прогнозирования на один или два уровня. Часто применяется, когда ряд не имеет
ярко выраженной тенденции развития. С помощью этого метода прогноз можно определить по следующей формуле:

\begin{equation}
	y_{t} = \frac{\sum_{i=1}^{n} y_i}{n},
\end{equation}

где $y_{i}$ — $i$-е фактическое значение исследуемого явления, $y_{t}$ — прогнозное значение, 
$n$ — число уровней.

\subsection{Метод скользящей средней}

Метод скользящей средней является одним из самых простых методов механического сглаживания, однако 
также применяется при решении задач прогнозирования [4]. Сначала для временного ряда $y1, y2, y3,\ldots,yn$ 
определяется интервал сглаживания $m, m<n$. Для первых m уровней временного ряда вычисляется их 
среднее арифметическое; это будет сглаженное значение уровня ряда, находящегося в середине интервала 
сглаживания. Затем интервал сглаживания сдвигается на один уровень вправо, повторяется вычисление 
среднего арифметического и т.д. 

В результате получается $n - m + 1$ сглаженных значений. Основными недостатками являются ограниченная
область применения метода (подходит только для рядов с линейной тенденцией) и невозможность сгладить 
первые m значений ряда.

\subsection{Метод экспоненциального сглаживания}

Метод экспоненциального сглаживания наиболее эффективен при разработке среднесрочных прогнозов. 
Имеет достаточно простую процедуру вычислений и возможность учёта весов исходной информации. 
Вычисляется по формуле:

\begin{equation}
	U_{t} = \alpha \cdot y_{t-1} + (1 - \alpha) \cdot U_{t-1},
\end{equation}

где $t-1$ — период перед прогнозным, $t$ — прогнозный период, $U_{t}$ — прогнозируемый показатель,
$\alpha$ — параметр сглаживания, $y_{t-1}$ — фактическое значение исследуемого показателя за период, 
предшествующий прогнозному, $U_{t-1}$ — экспоненциально взвешенная средняя для периода, 
предшествующего прогнозному.

Значение $\alpha$ лежит в диапазоне $[0, 1]$. От величины $\alpha$ зависит, как быстро снижается 
вес влияния предшествующих наблюдений. Чем больше $\alpha$, тем меньше сказывается влияние 
предшествующих периодов.

Автор данного метода, профессор Браун, предложил определять величину $\alpha$ в зависимости от интервала
сглаживания. Тогда $\alpha$ вычисляется по формуле:

\begin{equation}
	\alpha = \frac{2}{n + 1},
\end{equation}

\subsection{Метод Хольта}

Для метода Хольта требуется ввести характеристику временного ряда — тренд. Тренд – некоторая аналитическая функция, которая связывает единым 
«законом движения» все последовательные уровни временного ряда [5]. 

Метод Хольта используется для прогнозирования временных рядов, когда имеется тенденция к росту или 
падению значений временного ряда, а также для рядов, когда данных недостаточно для выделения сезонного
периода.

Для реализации метода вводится два коэффициента сглаживания: $\alpha$ – коэффициент сглаживания ряда 
и $\beta$ – тренда.

Рассчитывается экспоненциально–сглаженный ряд: 
\begin{equation}
	L_{t} = \alpha \cdot y_{t} + (1 - \alpha) \cdot (L_{t-1} - T_{t-1}),
\end{equation}

где $L_{t}$ – сглаженная величина на текущий период, $L_{t-1}$ – экспоненциально сглаженное значение 
на предыдущий период, $T_{t-1}$ – значение тренда за предыдущий период.

Определяется значение тренда: 
\begin{equation}
	T_{t} = \beta \cdot (L_{t} - L_{t-1}) + (1 - \beta) \cdot T_{t-1}.
\end{equation}

Прогнозное значение вычисляется по формуле:
\begin{equation}
	y_{t+p} = L_{t} + p \cdot T_{t},
\end{equation}
где $y_{t+p}$ – прогноз на период $p$, $p$ – порядковый номер периода, на который делается прогноз.
